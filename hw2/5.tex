\documentclass[11pt]{article}
\usepackage{subfigure}
\usepackage{color}
\usepackage{url}
\usepackage{graphicx}
\usepackage{fullpage}
\usepackage[english]{babel}
\usepackage{amssymb}
\usepackage{amsmath}
\usepackage{fancyhdr}
\usepackage{hyperref}
\usepackage{algorithmic}
\usepackage{algorithm}
\usepackage{enumerate}
\usepackage{mdframed}

\DeclareMathOperator*{\argmin}{argmin}
\newcommand*{\argminl}{\argmin\limits}


\begin{document}
\begin{center}
%---------------------------------------------------------------------------------------
%---------------------------------Header------------------------------------------------
%---------------------------------------------------------------------------------------

\framebox{\parbox{6.5in}{
{\bf{STATS 315B: Data Mining, Spring 2016}}\\
{\bf Homework 2, Due 5/22/2016}\\
{\bf Completed by: Henry Neeb, Christopher Kurrus, Yash Vyas and Tyler Chase}
}}
\ \\
\end{center}

%---------------------------------------------------------------------------------------
%---------------------------------Answer------------------------------------------------
%---------------------------------------------------------------------------------------

\section*{Problem 5}

\vspace{5 mm}
\noindent
Given our additive function $F(x) = F_{l}(z_{l}) + F_{\backslash l}(z_{\backslash l})$
 on our subsets, we can represent the partial dependence of $F(x)$ on $z_{l}$ as
  $E_{z_{\backslash l}}(F(z_{l}, z_{\backslash l}))$ which we can approximate with 
  $\frac{1}{N} \sum\limits_{i=1}^N F(z_{l}, z_{i\backslash l})$ where $N$ is the
   number of elements in $z_{\backslash l}$.  This is from Elements of Statistical
    Learning 10.47 and 10.48 (pg. 369).  

\noindent
Now we can substitute in our additive function, getting 
$\frac{1}{N} \sum\limits_{i=1}^N F_{l}(z_{l}) + F_{\backslash l}(z_{i\backslash l})$  

\noindent
From this we can split the summation: 
$\frac{1}{N} \sum\limits_{i=1}^N F_{l}(z_{l}) + \frac{1}{N} \sum\limits_{i=1}^N 
F_{\backslash l}(z_{i\backslash l})$ 

\noindent
Because $F_{l}(z_{l})$ doesn't depend on i, we arrive at $F_{l}(z_{l}) + \frac{1}{N}
 \sum\limits_{i=1}^N F_{\backslash l}(z_{i\backslash l})$ which is $F_{l}(z_{l})$ 
 plus an additive constant, as desired.

\vspace{5 mm}
\noindent
Now considering the dependence ignoring the additional variables $z_{\backslash l}$, 
we have $E[F(x)|z_{l}]$. If we assume some density functions $p_{z_{l}}$ and
 $p_{z_{\backslash l}}$, we can then say $E[F(x)|z_{l}] = \int p_{z_{l}}F_{l}(z_{l})
 d_{z_{l}} + \int p_{z_{\backslash l}}F_{\backslash l}(z_{\backslash l})d_{z_{\backslash l}}$
 using the additive property of $F(x)$.  Now we know that this is all given $z_{l}$, so 
 the first term $\int p_{z_{l}}F_{l}(z_{l})d_{z_{l}}$ will simply evaluate to some number,
 but the second term $\int p_{z_{\backslash l}}F_{\backslash l}(z_{\backslash l})
 d_{z_{\backslash l}}$ will be some function not dependent on $z_{l}$. Therefore our
  dependence ignoring the additional variables $z_{\backslash l}$ will not always 
  be $F_{l}(z_{l})$ plus an additive constant.

\vspace{5 mm}
\noindent
 The partial dependence of $F(x)$ on $z_{l}$ and the partial dependence of $F(x)$ on 
 $z_{l}$ ignoring the additional variables $z_{\backslash l}$ will be equivalent
  when the two subsets $z_{l}$ and $z_{\backslash l}$ are independent.

\end{document}