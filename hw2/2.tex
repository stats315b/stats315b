\documentclass[11pt]{article}
\usepackage{subfigure}
\usepackage{color}
\usepackage{url}
\usepackage{graphicx}
\usepackage{fullpage}
\usepackage[english]{babel}
\usepackage{amssymb}
\usepackage{amsmath}
\usepackage{fancyhdr}
\usepackage{hyperref}
\usepackage{algorithmic}
\usepackage{algorithm}
\usepackage{enumerate}
\usepackage{mdframed}
\usepackage{mathrsfs}


\begin{document}
\begin{center}
%---------------------------------------------------------------------------------------
%---------------------------------Header------------------------------------------------
%---------------------------------------------------------------------------------------

\framebox{\parbox{6.5in}{
{\bf{STATS 315B: Data Mining, Spring 2016}}\\
{\bf Homework 2, Due 5/22/2016}\\
{\bf Completed by: Henry Neeb, Christopher Kurrus, Tyler Chase, and Yash Vyas}
}}
\ \\
\end{center}

\section*{Problem 2}

\vspace{5 mm}
\noindent
Linear regression uses least squares as the error function. When the number of predictors is
greater than the number of observations in the training data, we have multiple perfect fits for the
model and therefore, infinite solutions for the coefficient estimates of the least squares. Thus, the model
would have a huge variance.

Regularization is necessary to find the unique coefficient estimates when the number of observations
is less than the number of predictors. It also helps in cases when there is a high collinearity/correlation 
among the predictors in the model. Another case where regularisation helps is when there exist multiple 
noisy variables in the data. In such a case, regularization decreases the coefficient estimates of the noisy 
variables. 

Regularization is disadvantageous when the number predictors is extremely small. In this case, one would 
ideally like to pursue feature augmentation. But regularisation does not support feature augmentation. 
Another example where regularization fails is when we have a perfect line/plane that fits the model. 
Generally, regularization fails when the bias introduced by it does not lead to drastic decrease in the 
variance of prediction. 

During boosting, we generate a linear combination of all possible trees. Sparsity is a reasonable assumption 
in this situation because we generate multiple trees, and only a fraction of the trees capture the signal in the data. 

Most of the trees generated in boosting do not influence the outcome variable. We simply do not have 
enough data to estimate a large number of parameters (all possible trees in the boosting context). Hence, 
sparsity assumption always holds true in boosting.


\vspace{5 mm}
\noindent


\end{document}