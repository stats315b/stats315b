\documentclass[11pt]{article}
\usepackage{subfigure}
\usepackage{color}
\usepackage{url}
\usepackage{graphicx}
\usepackage{fullpage}
\usepackage[english]{babel}
\usepackage{amssymb}
\usepackage{amsmath}
\usepackage{fancyhdr}
\usepackage{hyperref}
\usepackage{algorithmic}
\usepackage{algorithm}
\usepackage{enumerate}
\usepackage{mdframed}
\usepackage{mathrsfs}


\begin{document}
\begin{center}
%---------------------------------------------------------------------------------------
%---------------------------------Header------------------------------------------------
%---------------------------------------------------------------------------------------

\framebox{\parbox{6.5in}{
{\bf{STATS 315B: Data Mining, Spring 2016}}\\
{\bf Homework 1, Due 4/28/2016}\\
{\bf Completed by: Henry Neeb, Christopher Kurrus, and Tyler Chase}
}}
\ \\
\end{center}

\section*{Problem 3}

$$Target Function: F^* = argmin_FR(F)$$
$$Risk Function: R(F) = E_{XY}L(Y, F(\underline{X})$$
$$Loss Criterion: L(Y,F(\underline{X}))$$

Y is the actual output and \underline{X} is a vector of predictors. The target function is not always an accurate function for prediction. Even if the target function predicts accurately with cross vallidation on the training set, the correlations between the output and predictors could change as a function of time. This would result in inaccurate future predicitons. 

It could also be inaccurate if there is little to no correlation between the predictors and output to begin with; or if the signal to noise is too low. 

\end{document}