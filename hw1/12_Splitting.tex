\documentclass[11pt]{article}
\usepackage{subfigure}
\usepackage{color}
\usepackage{url}
\usepackage{graphicx}
\usepackage{fullpage}
\usepackage[english]{babel}
\usepackage{amssymb}
\usepackage{amsmath}
\usepackage{fancyhdr}
\usepackage{hyperref}
\usepackage{algorithmic}
\usepackage{algorithm}
\usepackage{enumerate}
\usepackage{mdframed}

\DeclareMathOperator*{\argmin}{argmin}
\newcommand*{\argminl}{\argmin\limits}


\begin{document}
\begin{center}
%-------------------------------------------------------------------------------
%---------------------------------Header----------------------------------------
%--------------------------------------------------------------------------------

\framebox{\parbox{6.5in}{
{\bf{STATS 315B: Data Mining, Spring 2016}}\\
{\bf Homework 1, Due 4/28/2016}\\
{\bf Completed by: Henry Neeb, Christopher Kurrus, and Tyler Chase}
}}
\ \\
\end{center}

%-------------------------------------------------------------------------------
%---------------------------------Answer----------------------------------------
%-------------------------------------------------------------------------------

\vspace{5 mm}
\noindent
{\bf Treat Node as Terminal:} Treating the node as terminal means we predict on 
the majority at the node with the missing value. Why this is ok:

\begin{itemize}
\item Surrogate might be a bad predictor. Weak correlation might split 
erroneously, and we have very little control over that.
\item With homogeneous data (i.e. you have roughly the same amount of different 
types of responses) it might be ok, because as you pull out other types of 
responses and you're left with a dominating set of one response, it is at this 
point very likely that you will predict the majority anyway.
\end{itemize}

Why it might be bad:

\begin{itemize}
\item Shear majority rules without going through all the splits might be 
problematic depending on the data you have collected. For example, let's say 
you were building a decision tree that predicts a rare disease. Your tree will 
be susceptible to always deciding on a patient not having the disease when 
there is missing data because most likely your training data will have mostly 
those without the disease.
\item We know from tree construction, the previous split you have taken may not 
be useful by itself. Predicting at the terminal node does not allow other 
features from the data to help with prediction.
\item Majority rules at the node where there are roughly equal amounts of the 
same responses is not a compelling prediction.
\end{itemize}

{\bf Split by Majority:} Why this is ok:

\begin{itemize}
\item 
\end{itemize}

Why it might be bad:

\begin{itemize}
\item 
\end{itemize}


\end{document}